\documentclass[14pt,a4paper]{extarticle}
\usepackage[utf8]{inputenc}
\usepackage[T2A]{fontenc}
\usepackage[english,russian]{babel}
\usepackage{amsmath,amsfonts,amssymb,amsthm,mathtools,tikz,subcaption,float}
\setlength{\topmargin}{-1.5cm}
\setlength{\textheight}{23.3cm}
\setlength{\oddsidemargin}{-0.5cm}
\setlength{\textwidth}{17.0cm}
\newcommand{\V}[1]{\mbox{\boldmath$#1$}}
\newcommand{\T}[1]{\mbox{\bf#1}}
\DeclareUnicodeCharacter{2061}{}
\usepackage{graphicx}
\graphicspath{ {./images} }
\usepackage{wrapfig}
\textheight=240mm
\usepackage{mathtools}
\pagestyle{empty}
\setlength\parindent{12.5mm}
\usepackage{tocloft}
\renewcommand{\cfttoctitlefont}{\normalsize\bfseries}
\renewcommand{\cftsecleader}{\cftdotfill{\cftdotsep}}%простые разреженные отточия
\renewcommand{\cftsecfont}{\normalsize}
\renewcommand\cftsecpagefont{\normalfont}
\renewcommand{\baselinestretch}{1.25}
\usepackage{titlesec}
\titleformat{\section}
  {\normalfont\normalsize\bfseries}{\thesection}{1em}{}
\titleformat{\subsection}
  {\normalfont\normalsize\bfseries}{\thesection}{1em}{}
\usepackage{tocloft}
\renewcommand{\cfttoctitlefont}{\normalsize\bfseries}
\renewcommand{\cftsecleader}{\cftdotfill{\cftdotsep}}%простые разреженные отточия
\renewcommand{\cftsecfont}{\normalsize}
\renewcommand\cftsecpagefont{\normalfont}
\newcommand{\stone}[3]{\filldraw[color=black, fill=#3, very thick](#1,#2) circle(0.45);}
\newcommand{\territory}[3]{\filldraw[color=black, fill=#3, very thick](#1, #2) circle(0.125);}
\newcommand{\stonelabel}[4]{\node[text=#3] at (#1,#2) {#4};}

\begin{document}



\thispagestyle{empty}
\begin{center}
\small{МИНИСТЕРСТВО НАУКИ И ВЫСШЕГО ОБРАЗОВАНИЯ}\par
\small{РОССИЙСКОЙ ФЕДЕРАЦИИ}\par
\small{ФЕДЕРАЛЬНОЕ ГОСУДАРСТВЕННОЕ АВТОНОМНОЕ ОБРАЗОВАТЕЛЬНОЕ УЧРЕЖДЕНИЕ ВЫСШЕГО ОБРАЗОВАНИЯ}\par
\small{<<ЮЖНЫЙ ФЕДЕРАЛЬНЫЙ УНИВЕРСИТЕТ>>}\par
\normalsize{Институт математики механики и компьютерных наук им. И.\,И. Воровича}\par
\end{center}


\vspace*{5.0em plus 0.2em minus 0.2em}

\begin{center}
Методическое пособие по игре в го
\end{center}

\begin{center}
для начинающих.
\end{center}

\begin{center}
Часть 1?
\end{center}

\vspace*{5.0em plus 0.2em minus 0.2em}
\begin{flushright}
 Выполнили:\\
	...
\end{flushright}






\vspace*{\fill}

\begin{center}
г. Ростов-на-Дону, 2023
\end{center}

\newpage
\thispagestyle{empty}
\tableofcontents


\newpage

\section*{Предисловие.}\addcontentsline{toc}{section}{Предисловие}

\subsection*{Что такое го?}\addcontentsline{toc}{subsection}{Что такое го?}
\

Го - не только самая древняя, но и самая сложная среди всех стратегических игр. Количество возможных позиций в обычной партии примерно равно $10^{171}$. Правила го настолько просты, что даже ребёнок может освоить их за полчаса, однако для достижения мастерства понадобятся годы. 

\subsection*{История го.}\addcontentsline{toc}{subsection}{История го}
\

Истоки игры го берут своё начало в Китае, откуда она распространилась на Японию и Корею. Мировую известность приобрёл её японский вариант в конце XIX — начале XX века. Эта игра стала частью восточной философии, искусства и литературы, а в наше время заняла свое место в бизнес-стратегии, когнитивной психологии, математике, исследованиях в области искусственного интеллекта и на полках интеллектуалов всего мира.

По историческим данным игре го сейчас не меньше 2,5 тыс. лет. Существует предположение о том, что го около 4-5 тыс. лет. Возникла эта игра в древнем Китае, в районе Тибета. Одной из самых распространённых версий о происхождении го является легенда о том, что сначала люди использовали эту игру для гадания. Отсюда и происходит особый, астрологический символизм ключевых «звёздных» пунктов (хоси). Со временем мистика исчезла из го. Она стала игрой с полной информацией и перешла в разряд развлечений интеллектуальной элиты.

В различных регионах игра го была известна под разными названиями: в Древнем Китае за ней закрепилось название «вэйци», в Корее – «бадук», во Вьетнаме – «ко-вай», в Тибете – «минг манг» и «джялпо», в Монголии – «дёрвёлц». «Го» является японским названием. В дореволюционной России, куда игру привезли послы китайской дипломатической миссии, её называли «облавные шашки», что, однако, не совсем соответствует цели игры, поскольку соперники стремятся не захватить камни противника, а оградить на доске наибольшую территорию. 

Когда в начале XX века игра проникла (просочилась) на Запад, за нею закрепилось японское название го. В английском языке принято писать его с заглавной буквы, чтобы отличать от глагола «go». Становление современного го принято связывать, в основном, с Японией, где игра пережила сильнейший подъём в XV веке, оформилась как искусство и спортивная дисциплина.

\subsection*{Цель игры.}\addcontentsline{toc}{subsection}{Цель игры}
\ 

В игре участвуют два соперника. Их цель - разделить доску на владения. При этом каждый игрок стремится в процессе раздела получить контроль над большей частью доски, чем партнер. По очереди выставляя на доску камни, противники создают неповторимый черно-белый узор, шаг за шагом выстраивая свои владения, окружая пустоту на доске. Более того, в процессе игры противники могут окружать и захватывать камни друг друга. Тем самым игрок увеличивает свои приобретения, одновременно уменьшая завоевания противника.

Секрет успеха в игре - постоянное стремление к гармонии. Любая крайность губительна. Чрезмерная агрессия так же быстро ведет к поражению, как и излишняя робость. Не менее важен и этикет, который предписывает вежливое, внимательное отношение к партнеру.

\newpage

\section*{Основы игры.}\addcontentsline{toc}{section}{Основы игры}

\subsection*{Правила игры.}\addcontentsline{toc}{subsection}{Правила игры}
\

Для игры понадобятся доска $9\times9$, $13\times13$, $15\times15$, или $19\times19$ (доски поменьше обычно подходят для начинающих игроков) и два набора камней разных цветов (обычно чёрного и белого).

Рассмотрм 5 базовых правил го.
\\


\textit{Правило 1.} О постановке камней на доску.

Чёрные и белые ходят по очереди. Начинают партию чёрные с пустой доски. Каждым своим ходом игроки могут поставить камень своего цвета на доску или сказать "пас". Камни ставятся на пересечение линий, называемое пунктом, и не могут быть потом передвинуты (лишь только сняты, но об этом позже).

\begin{figure}[h]
\centering
\begin{tikzpicture}
\draw [step=1,gray] (0,0) grid (5.5,5.5);
\draw[black,ultra thick](0,0) -- (5.5,0);
\draw[black,ultra thick](0,0) -- (0,5.5);
\filldraw[color=black, fill=black, very thick](4,3) circle (0.45);
\filldraw[color=black, fill=white, very thick](2,1) circle (0.45);
\filldraw[color=black, fill=black, very thick](1,1) circle (0.45);
\filldraw[color=black, fill=white, very thick](2,4) circle (0.45);
\filldraw[color=black, fill=black, very thick](0,4) circle (0.45);
\end{tikzpicture}
\caption{Просто какая-то расстановка камней на фрагменте доски.} \label{1}
\end{figure}

\textit{Правило 2.} О дыханиях и снятии камней с доски.

Под дыханиями (или степенями свободы, или дамэ) камня будем понимать все соседние по стороне пункты, не занятые камнями.

\newpage
\begin{figure}[h!]
\centering
\begin{tikzpicture}
\draw [step=1,gray] (0,0) grid (5.5,5.45);
\draw[black,ultra thick](0,0) -- (5.5,0);
\draw[black,ultra thick](0,0) -- (0,5.5);
\filldraw[color=black, fill=black, very thick](0,0) circle (0.45);
\filldraw[color=black, fill=black, thick](0,1) circle (0.125);
\filldraw[color=black, fill=black, thick](1,0) circle (0.125);
\filldraw[color=black, fill=black, very thick](4,0) circle (0.45);
\filldraw[color=black, fill=black, thick](5,0) circle (0.125);
\filldraw[color=black, fill=black, thick](3,0) circle (0.125);
\filldraw[color=black, fill=black, thick](4,1) circle (0.125);
\filldraw[color=black, fill=black, very thick](3,3) circle (0.45);
\filldraw[color=black, fill=black, thick](3,2) circle (0.125);
\filldraw[color=black, fill=black, thick](3,4) circle (0.125);
\filldraw[color=black, fill=black, thick](2,3) circle (0.125);
\filldraw[color=black, fill=black, thick](4,3) circle (0.125);
\end{tikzpicture}
\caption{Дыхания камней.} \label{2}
\end{figure}

Говорят, что камни одного цвета образуют группу камней, если они соединены по стороне. Тогда будем считать, что у группы камней общие дыхания.

\begin{figure}[h!]
    \centering
    
    \begin{subfigure}[t]{0.35\textwidth}
    \begin{tikzpicture}
    \draw [step=1,gray] (0,0) grid (5.5,5.45);
    \draw[black,ultra thick](0,0) -- (5.5,0);
    \draw[black,ultra thick](0,0) -- (0,5.5);
    \filldraw[color=black, fill=black, very thick](3,3) circle (0.45);
    \filldraw[color=black, fill=black, very thick](3,4) circle (0.45);
    \filldraw[color=black, fill=black, very thick](3,2) circle (0.45);
    \filldraw[color=black, fill=black, thick](2,3) circle (0.125);
    \filldraw[color=black, fill=black, thick](2,4) circle (0.125);
    \filldraw[color=black, fill=black, thick](2,2) circle (0.125);
    \filldraw[color=black, fill=black, thick](4,3) circle (0.125);
    \filldraw[color=black, fill=black, thick](4,2) circle (0.125);
    \filldraw[color=black, fill=black, thick](4,4) circle (0.125);
    \filldraw[color=black, fill=black, thick](3,1) circle (0.125);
    \filldraw[color=black, fill=black, thick](3,5) circle (0.125);
    \end{tikzpicture}
    \caption{Дыхания группы камней.} 
    \label{3a}
    \end{subfigure}
    \hfill
    \begin{subfigure}[t]{0.32\textwidth}
    \begin{tikzpicture}
    \draw [step=1,gray] (0,0) grid (4.5,5.5);
    \draw[black,ultra thick](0,0) -- (4.5,0);
    \draw[black,ultra thick](0,0) -- (0,5.5);
    \filldraw[color=black, fill=black, very thick](0,3) circle (0.45);
    \filldraw[color=black, fill=black, very thick](0,4) circle (0.45);
    \filldraw[color=black, fill=black, very thick](1,3) circle (0.45);
    \filldraw[color=black, fill=black, thick](0,2) circle (0.125);
    \filldraw[color=black, fill=black, thick](1,2) circle (0.125);
    \filldraw[color=black, fill=black, thick](2,3) circle (0.125);
    \filldraw[color=black, fill=black, thick](1,4) circle (0.125);
    \filldraw[color=black, fill=black, thick](0,5) circle (0.125);
    \end{tikzpicture}
    \caption{Дыхания группы камней.} 
    \label{3b}
    \end{subfigure}
    \hfill
    \begin{subfigure}[t]{0.27\textwidth}
    \begin{tikzpicture}
    \draw [step=1,gray] (0,0) grid (3.5,5.5);
    \draw[black,ultra thick](0,0) -- (3.5,0);
    \draw[black,ultra thick](0,0) -- (0,5.5);
    \filldraw[color=black, fill=black, very thick](1,3) circle (0.45);
    \filldraw[color=black, fill=black, very thick](2,4) circle (0.45);
    \filldraw[color=black, fill=black, very thick](2,2) circle (0.45);
    \end{tikzpicture}
    \caption{Не группа камней! Это просто 3~камня.} 
    \label{3c}
    \end{subfigure}
    \caption{}
\end{figure}

Если камень или группа камней лишаетяся всех дыханий, то эти камни снимаются с доски и попадают во вражеский плен.
\newpage
\begin{figure}[h!]
\centering
\begin{subfigure}[t]{0.4\textwidth}
\begin{tikzpicture}
\draw [step=1,gray] (0,0) grid (5.5,5.5);
\draw[black,ultra thick](0,0) -- (5.5,0);
\draw[black,ultra thick](0,0) -- (0,5.5);

\filldraw[color=black, fill=black!60, very thick](3,3) circle (0.45);
\filldraw[color=black, fill=white, very thick](3,2) circle (0.45);
\filldraw[color=black, fill=white, very thick](3,4) circle (0.45);
\filldraw[color=black, fill=white, very thick](2,3) circle (0.45);
\filldraw[color=black, fill=white, very thick](4,3) circle (0.45);
\end{tikzpicture}
\caption{}
\label{4a}
\end{subfigure}
\hfill
\begin{subfigure}[t]{0.4\textwidth}
    \begin{tikzpicture}
    \draw [step=1,gray] (0,0) grid (5.5,5.5);
    \draw[black,ultra thick](0,0) -- (5.5,0);
    \draw[black,ultra thick](0,0) -- (0,5.5);
    \filldraw[color=black, fill=black!60, very thick](3,3) circle (0.45);
    \filldraw[color=black, fill=black!60, very thick](3,4) circle (0.45);
    \filldraw[color=black, fill=black!60, very thick](3,2) circle (0.45);
    \filldraw[color=black, fill=white, very thick](2,3) circle (0.45);
    \filldraw[color=black, fill=white, very thick](2,4) circle (0.45);
    \filldraw[color=black, fill=white, very thick](2,2) circle (0.45);
    \filldraw[color=black, fill=white, very thick](4,3) circle (0.45);
    \filldraw[color=black, fill=white, very thick](4,2) circle (0.45);
    \filldraw[color=black, fill=white, very thick](4,4) circle (0.45);
    \filldraw[color=black, fill=white, very thick](3,1) circle (0.45);
    \filldraw[color=black, fill=white, very thick](3,5) circle (0.45);
    \end{tikzpicture}
    \caption{}
\label{4b}
    \end{subfigure}
\caption{Убитые камни или группы камней.}
\label{4}
\end{figure}

\textit{Правило 3.} О запрещённом ходе и исключении и этого правила.

Представим, какая позиция осталась на рис \ref{4a} после снятия чёрного камня. Поймём, что если вдруг чёрные захотят поставить камень на то же место, то он уже будет лишён всех дыханий.
Согласно правилам го, запрещено делать самоубийсвенный ход.

\begin{figure}[h!]
\centering
\begin{tikzpicture}
\draw [step=1,gray] (0,0) grid (5.5,5.5);
\draw[black,ultra thick](0,0) -- (5.5,0);
\draw[black,ultra thick](0,0) -- (0,5.5);

\filldraw[color=black, fill=black, very thick](3,3) circle (0.125);
\filldraw[color=black, fill=white, very thick](3,2) circle (0.45);
\filldraw[color=black, fill=white, very thick](3,4) circle (0.45);
\filldraw[color=black, fill=white, very thick](2,3) circle (0.45);
\filldraw[color=black, fill=white, very thick](4,3) circle (0.45);
\end{tikzpicture}

\caption{Пойдя в выделенный пункт, чёрные сразу лишат себя дыханий.}
\label{5}
\end{figure}

К тому же запрещено делать ход, отбирающий у своей группы посленднюю степень свободы. На рис. \ref{6} при попытке чёрных поёти в выделенный пункт, образуется позиция, как на рис. \ref{4b}, что, как мы видим, лишает группу всех оставшихся чтепеней свободы.
\newpage
\begin{figure}[h!]
\centering
\begin{tikzpicture}
\draw [step=1,gray] (0,0) grid (5.5,5.5);
    \draw[black,ultra thick](0,0) -- (5.5,0);
    \draw[black,ultra thick](0,0) -- (0,5.5);
    \filldraw[color=black, fill=black, very thick](3,3) circle (0.45);
    \filldraw[color=black, fill=black, very thick](3,4) circle (0.45);
    \filldraw[color=black, fill=black, very thick](3,2) circle (0.125);
    \filldraw[color=black, fill=white, very thick](2,3) circle (0.45);
    \filldraw[color=black, fill=white, very thick](2,4) circle (0.45);
    \filldraw[color=black, fill=white, very thick](2,2) circle (0.45);
    \filldraw[color=black, fill=white, very thick](4,3) circle (0.45);
    \filldraw[color=black, fill=white, very thick](4,2) circle (0.45);
    \filldraw[color=black, fill=white, very thick](4,4) circle (0.45);
    \filldraw[color=black, fill=white, very thick](3,1) circle (0.45);
    \filldraw[color=black, fill=white, very thick](3,5) circle (0.45);
\end{tikzpicture}

\caption{Пойдя в выделенный пункт, чёрные отнимут последнее дыхание у группы.}
\label{6}
\end{figure}
Теперь предположим, что в процессе игры, изображённой на рис. \ref{5} позиция преобразовалась к диаграмме \ref{7}. В этой, по правилам, уже можно сделать ход чёрным в выделенный пункт, т.к. белый камень 1 остаются без единой степени свободы. Иначе говоря, самоубийсвенный ход делать можно, если притом вражеский(е) камень(камни) лишаются последней степени свободы. И т.к. у поставленного чёрного камня уже будет дыхание, то он (пока что) будет поствлен.

\begin{figure}[h!]
\centering
\begin{tikzpicture}
\draw [step=1,gray] (0,0) grid (5.5,5.5);
\draw[black,ultra thick](0,0) -- (5.5,0);
\draw[black,ultra thick](0,0) -- (0,5.5);

\filldraw[color=black, fill=black, very thick](3,3) circle (0.125);
\filldraw[color=black, fill=white, very thick](3,2) circle (0.45);
\node[] at (3,2) {1};
\filldraw[color=black, fill=white, very thick](3,4) circle (0.45);
\filldraw[color=black, fill=white, very thick](2,3) circle (0.45);
\filldraw[color=black, fill=white, very thick](4,3) circle (0.45);
\filldraw[color=black, fill=black, very thick](2,2) circle (0.45);
\filldraw[color=black, fill=black, very thick](3,1) circle (0.45);
\filldraw[color=black, fill=black, very thick](4,2) circle (0.45);
\end{tikzpicture}

\caption{Пойдя в выделенный пункт, чёрные лишат сначала белый камень 1 дыхания.}
\label{7}
\end{figure}

\textit{Правило 4.} Правило ко.

Пусть в позиции, изображённой на рис. \ref{7}, чёрные взяли белый камень, тогда заметим, что у этого чёрного камня осталось одно дыхание, и белые могут вернуть сразу же камень назад. И мы вновь вернулись к начальной позиции.

\begin{figure}[h!]
\centering

\begin{subfigure}{0.4\textwidth}
\begin{tikzpicture}
    \draw [step=1,gray] (0,0) grid (7.5,7.5);
    \draw[black,ultra thick](0,0) -- (7.5,0);
    \draw[black,ultra thick](0,0) -- (0,7.5);
    \filldraw[color=black, fill=black, very thick](3,3) circle (0.45);
    \node[text=white] at (3,3) {1};
    \filldraw[color=black, fill=white, very thick](3,4) circle (0.45);
    \filldraw[color=black, fill=white, very thick](2,3) circle (0.45);
    \filldraw[color=black, fill=white, very thick](4,3) circle (0.45);
    \filldraw[color=black, fill=black, very thick](2,2) circle (0.45);
    \filldraw[color=black, fill=black, very thick](3,1) circle (0.45);
    \filldraw[color=black, fill=black, very thick](4,2) circle (0.45);
\end{tikzpicture}
\end{subfigure}
\hfill
\begin{subfigure}{0.4\textwidth}
\begin{tikzpicture}
    \draw [step=1,gray] (0,0) grid (7.5,7.5);
    \draw[black,ultra thick](0,0) -- (7.5,0);
    \draw[black,ultra thick](0,0) -- (0,7.5);
    \filldraw[color=black, fill=white, very thick](3,2) circle (0.45);
    \node[] at (3,2) {2};
    \filldraw[color=black, fill=white, very thick](3,4) circle (0.45);
    \filldraw[color=black, fill=white, very thick](2,3) circle (0.45);
    \filldraw[color=black, fill=white, very thick](4,3) circle (0.45);
    \filldraw[color=black, fill=black, very thick](2,2) circle (0.45);
    \filldraw[color=black, fill=black, very thick](3,1) circle (0.45);
    \filldraw[color=black, fill=black, very thick](4,2) circle (0.45);
\end{tikzpicture}
\end{subfigure}
\caption{Повторение позиции.}
\label{8}
\end{figure}

Для исключения бесконечного повторения позиции существует правило ко.
Согласно ему, никакая позиция на доске в партии после непасового хода не может повториться.

Например, в приведённой позиции партия модет пойти следующим образом, как на рис. \ref{9}

\begin{figure}[H]
\centering

\begin{subfigure}[t]{0.4\textwidth}
\begin{tikzpicture}[xscale=0.8,yscale=0.8]
    \draw [step=1,gray] (0,0) grid (7.5,7.5);
    \draw[black,ultra thick](0,0) -- (7.5,0);
    \draw[black,ultra thick](0,0) -- (0,7.5);
    \filldraw[color=black, fill=black, very thick](3,3) circle (0.45);
    \node[text=white] at (3,3) {1};
    \filldraw[color=black, fill=white, very thick](3,4) circle (0.45);
    \filldraw[color=black, fill=white, very thick](2,3) circle (0.45);
    \filldraw[color=black, fill=white, very thick](4,3) circle (0.45);
    \filldraw[color=black, fill=black, very thick](2,2) circle (0.45);
    \filldraw[color=black, fill=black, very thick](3,1) circle (0.45);
    \filldraw[color=black, fill=black, very thick](4,2) circle (0.45);
\end{tikzpicture}
\caption{Чёрные сняли камень.}
\label{9a}
\end{subfigure}
\hfill
\begin{subfigure}[t]{0.4\textwidth}
\begin{tikzpicture}[xscale=0.8,yscale=0.8]
    \draw [step=1,gray] (0,0) grid (7.5,7.5);
    \draw[black,ultra thick](0,0) -- (7.5,0);
    \draw[black,ultra thick](0,0) -- (0,7.5);
    \filldraw[color=black, fill=black, very thick](3,3) circle (0.45);
    \filldraw[color=black, fill=white, very thick](6,6) circle (0.45);
    \node[] at (6,6) {2};
    \filldraw[color=black, fill=white, very thick](3,4) circle (0.45);
    \filldraw[color=black, fill=white, very thick](2,3) circle (0.45);
    \filldraw[color=black, fill=white, very thick](4,3) circle (0.45);
    \filldraw[color=black, fill=black, very thick](2,2) circle (0.45);
    \filldraw[color=black, fill=black, very thick](3,1) circle (0.45);
    \filldraw[color=black, fill=black, very thick](4,2) circle (0.45);
\end{tikzpicture}
\caption{Белые не могут снять в ответ, поэтому пошли в другое место.}
\label{9b}
\end{subfigure}

\begin{subfigure}[t]{0.4\textwidth}
\begin{tikzpicture}[xscale=0.8,yscale=0.8]
    \draw [step=1,gray] (0,0) grid (7.5,7.5);
    \draw[black,ultra thick](0,0) -- (7.5,0);
    \draw[black,ultra thick](0,0) -- (0,7.5);
    \filldraw[color=black, fill=black, very thick](3,3) circle (0.45);
    \filldraw[color=black, fill=white, very thick](6,6) circle (0.45);
    \filldraw[color=black, fill=black, very thick](7,5) circle (0.45);
    \node[text=white] at (7,5) {3};
    \filldraw[color=black, fill=white, very thick](3,4) circle (0.45);
    \filldraw[color=black, fill=white, very thick](2,3) circle (0.45);
    \filldraw[color=black, fill=white, very thick](4,3) circle (0.45);
    \filldraw[color=black, fill=black, very thick](2,2) circle (0.45);
    \filldraw[color=black, fill=black, very thick](3,1) circle (0.45);
    \filldraw[color=black, fill=black, very thick](4,2) circle (0.45);
\end{tikzpicture}
\caption{Пусть чёрным необходимо ответить на ход 2.}
\label{9c}
\end{subfigure}
\hfill
\begin{subfigure}[t]{0.4\textwidth}
\begin{tikzpicture}[xscale=0.8,yscale=0.8]
    \draw [step=1,gray] (0,0) grid (7.5,7.5);
    \draw[black,ultra thick](0,0) -- (7.5,0);
    \draw[black,ultra thick](0,0) -- (0,7.5);
    \filldraw[color=black, fill=white, very thick](3,2) circle (0.45);
    \filldraw[color=black, fill=white, very thick](6,6) circle (0.45);
    \filldraw[color=black, fill=black, very thick](7,5) circle (0.45);
    \node[] at (3,2) {4};
    \filldraw[color=black, fill=white, very thick](3,4) circle (0.45);
    \filldraw[color=black, fill=white, very thick](2,3) circle (0.45);
    \filldraw[color=black, fill=white, very thick](4,3) circle (0.45);
    \filldraw[color=black, fill=black, very thick](2,2) circle (0.45);
    \filldraw[color=black, fill=black, very thick](3,1) circle (0.45);
    \filldraw[color=black, fill=black, very thick](4,2) circle (0.45);
\end{tikzpicture}
\caption{Теперь белые могут снять камень, т.к. позиция изменилась.}
\label{9d}
\end{subfigure}

\begin{subfigure}[t]{0.4\textwidth}
\begin{tikzpicture}[xscale=0.8,yscale=0.8]
    \draw [step=1,gray] (0,0) grid (7.5,7.5);
    \draw[black,ultra thick](0,0) -- (7.5,0);
    \draw[black,ultra thick](0,0) -- (0,7.5);
    \filldraw[color=black, fill=white, very thick](3,2) circle (0.45);
    \filldraw[color=black, fill=white, very thick](6,6) circle (0.45);
    \filldraw[color=black, fill=black, very thick](7,5) circle (0.45);
    \filldraw[color=black, fill=black, very thick](6,5) circle (0.45);
    \node[text=white] at (6,5) {5};
    \filldraw[color=black, fill=white, very thick](3,4) circle (0.45);
    \filldraw[color=black, fill=white, very thick](2,3) circle (0.45);
    \filldraw[color=black, fill=white, very thick](4,3) circle (0.45);
    \filldraw[color=black, fill=black, very thick](2,2) circle (0.45);
    \filldraw[color=black, fill=black, very thick](3,1) circle (0.45);
    \filldraw[color=black, fill=black, very thick](4,2) circle (0.45);
\end{tikzpicture}
\caption{Чёрные не могут снять в ответ, поэтому пошли в другое место.}
\label{9e}
\end{subfigure}
\hfill
\begin{subfigure}[t]{0.4\textwidth}
\begin{tikzpicture}[xscale=0.8,yscale=0.8]
    \draw [step=1,gray] (0,0) grid (7.5,7.5);
    \draw[black,ultra thick](0,0) -- (7.5,0);
    \draw[black,ultra thick](0,0) -- (0,7.5);
    \filldraw[color=black, fill=white, very thick](3,2) circle (0.45);
    \filldraw[color=black, fill=white, very thick](6,6) circle (0.45);
    \filldraw[color=black, fill=black, very thick](7,5) circle (0.45);
    \filldraw[color=black, fill=black, very thick](6,5) circle (0.45);
    \filldraw[color=black, fill=white, very thick](3,3) circle (0.45);
    \node[] at (3,3) {6};
    \filldraw[color=black, fill=white, very thick](3,4) circle (0.45);
    \filldraw[color=black, fill=white, very thick](2,3) circle (0.45);
    \filldraw[color=black, fill=white, very thick](4,3) circle (0.45);
    \filldraw[color=black, fill=black, very thick](2,2) circle (0.45);
    \filldraw[color=black, fill=black, very thick](3,1) circle (0.45);
    \filldraw[color=black, fill=black, very thick](4,2) circle (0.45);
\end{tikzpicture}
\caption{Белые решили соединить камни, не среагировав на ход 5.}
\label{9f}
\end{subfigure}
\caption{Ко-борьба.}
\label{9}
\end{figure}

Здесь правило ко опрелелено неклассическим образом, в виду некоторых сложных случаев, которые пока что рассматриваться не будут.

\textit{Правило 5.} Территория, конец партии и подсчёт очков.

Партия заканчивается, быть может за исключением некоторого числа сложных и редких случаев, после двух последовательных пасов.

Под территорией будем понимать все не занятые камнями пункты, окружённые камнями одного цвета. Это определение дано не полностью и включает в себя ещё ряд важных аспектов, о которых будет сказано при решении задач. По большей части понятие территории являтся договорным, т.е. если оба игрока согласны, что определённая часть доски является территорией одного игрока, то так и будет считаться в итоговом результате. Пока что достаточно интуитивного понимания территории.

\begin{figure}[h]
\centering
\begin{tikzpicture}
    \draw [step=1,gray] (1,1) grid (11.5,11.5);
    \draw[black,ultra thick](1,1) -- (11.5,1);
    \draw[black,ultra thick](1,1) -- (1,11.5);
    \filldraw[color=black, fill=black, very thick](1,1) circle (0.45);
    \filldraw[color=black, fill=black, very thick](2,2) circle (0.45);
    \filldraw[color=black, fill=black, very thick](3,3) circle (0.45);
    \filldraw[color=black, fill=black, very thick](3,2) circle (0.45);
    \filldraw[color=black, fill=black, very thick](1,3) circle (0.45);
    \filldraw[color=black, fill=black, very thick](2,3) circle (0.45);
    \filldraw[color=black, fill=black, very thick](3,1) circle (0.45);
    \filldraw[color=black, fill=black, very thick](1,2) circle (0.125);
    \filldraw[color=black, fill=black, very thick](2,1) circle (0.125);

    \filldraw[color=black, fill=white, very thick](3,6) circle (0.45);
    \filldraw[color=black, fill=white, very thick](3,8) circle (0.45);
    \filldraw[color=black, fill=white, very thick](2,7) circle (0.45);
    \filldraw[color=black, fill=white, very thick](4,7) circle (0.45);
    \filldraw[color=black, fill=white, very thick](3,7) circle (0.125);

    \filldraw[color=black, fill=white, very thick](6,4) circle (0.125);
    \filldraw[color=black, fill=white, very thick](6,5) circle (0.125);
    \filldraw[color=black, fill=white, very thick](6,3) circle (0.125);
    \filldraw[color=black, fill=white, very thick](5,4) circle (0.45);
    \filldraw[color=black, fill=white, very thick](5,5) circle (0.45);
    \filldraw[color=black, fill=white, very thick](5,3) circle (0.45);
    \filldraw[color=black, fill=white, very thick](7,4) circle (0.45);
    \filldraw[color=black, fill=white, very thick](7,3) circle (0.45);
    \filldraw[color=black, fill=white, very thick](7,5) circle (0.45);
    \filldraw[color=black, fill=white, very thick](6,2) circle (0.45);
    \filldraw[color=black, fill=white, very thick](6,6) circle (0.45);

    \filldraw[color=black, fill=black, very thick](7,9) circle (0.45);
    \filldraw[color=black, fill=black, very thick](7,10) circle (0.45);
    \filldraw[color=black, fill=black, very thick](7,8) circle (0.45);
    \filldraw[color=black, fill=black, very thick](8,10) circle (0.45);
    \filldraw[color=black, fill=black, very thick](9,10) circle (0.45);
    \filldraw[color=black, fill=black, very thick](8,8) circle (0.45);
    \filldraw[color=black, fill=black, very thick](8,7) circle (0.45);
    \filldraw[color=black, fill=black, very thick](10,6) circle (0.45);
    \filldraw[color=black, fill=black, very thick](8,6) circle (0.45);
    \filldraw[color=black, fill=black, very thick](9,6) circle (0.45);
    \filldraw[color=black, fill=black, very thick](10,7) circle (0.45);
    \filldraw[color=black, fill=black, very thick](10,8) circle (0.45);
    \filldraw[color=black, fill=black, very thick](10,9) circle (0.45);
    \filldraw[color=black, fill=black, very thick](10,10) circle (0.45);
    \filldraw[color=black, fill=black, very thick](8,9) circle (0.125);
    \filldraw[color=black, fill=black, very thick](9,9) circle (0.125);
    \filldraw[color=black, fill=black, very thick](9,8) circle (0.125);
    \filldraw[color=black, fill=black, very thick](9,7) circle (0.125);
\end{tikzpicture}
\caption{Территории.}
\label{10}
\end{figure}




\newpage

\subsection*{Задачи.}\addcontentsline{toc}{subsection}{Задачи}

\


Определить какие группы мертвы, т.е. у них не осталось степеней свободы.

\noindent\textbf{1-I.}


\begin{figure}[h!]
\centering
\begin{tikzpicture}
    \draw [step=1,gray] (-0.5,-0.5) grid (7.5,7.5);
    \filldraw[color=black, fill=black, very thick](3,3) circle (0.45);
    \filldraw[color=black, fill=black, very thick](4,3) circle (0.45);
    \filldraw[color=black, fill=black, very thick](2,3) circle (0.45);
    \filldraw[color=black, fill=white, very thick](3,2) circle (0.45);
    \filldraw[color=black, fill=white, very thick](4,2) circle (0.45);
    \filldraw[color=black, fill=white, very thick](2,2) circle (0.45);
    \filldraw[color=black, fill=white, very thick](3,4) circle (0.45);
    \filldraw[color=black, fill=white, very thick](2,4) circle (0.45);
    \filldraw[color=black, fill=white, very thick](4,4) circle (0.45);
    \filldraw[color=black, fill=white, very thick](1,3) circle (0.45);
    \filldraw[color=black, fill=white, very thick](5,3) circle (0.45);
    \filldraw[color=black, fill=black, very thick](3,5) circle (0.45);
    \filldraw[color=black, fill=black, very thick](2,5) circle (0.45);
    \filldraw[color=black, fill=black, very thick](4,5) circle (0.45);
    \filldraw[color=black, fill=black, very thick](5,4) circle (0.45);
\end{tikzpicture}
\end{figure}

\noindent\textbf{2-I.}


\begin{figure}[h!]
	\centering
	\begin{tikzpicture}
		\draw[step=1, gray](0, 0) grid (8.5, 5.5);
		\draw[black, ultra thick](0, 0) -- (0, 5.5);
		\draw[black, ultra thick](0, 0) -- (8.5, 0);
		\stone{1}{3}{white}
		\stone{6}{3}{black}
		\stone{2}{2}{white}
		\stone{3}{2}{white}
		\stone{4}{2}{white}
		\stone{5}{2}{white}
		\stone{6}{2}{black}
		\stone{2}{1}{black}
		\stone{3}{1}{black}
		\stone{4}{1}{black}
		\stone{5}{1}{black}
		\stone{6}{1}{white}
		\stone{7}{1}{white}
		\stone{1}{0}{black}
		\stone{2}{0}{white}
		\stone{3}{0}{white}
		\stone{4}{0}{white}
		\stone{5}{0}{white}
		\stone{6}{0}{black}
		\stone{7}{0}{black}
	\end{tikzpicture}
\end{figure}

\newpage

Найдите ход, с помощью которого можно снять камень(ни) противника.


\noindent\textbf{3-I.} Ход чёрных.




\begin{figure}[h!]
	\centering
	\begin{tikzpicture}
		\draw[step=1, gray](2.5, 3.5) grid (9.5, 10.5);
		\stone{7}{9}{white}
		\stone{7}{8}{black}
		\stone{7}{7}{white}
		\stone{8}{7}{black}
		\stone{4}{6}{black}
		\stone{5}{6}{black}
		\stone{6}{6}{white}
		\stone{7}{6}{white}
		\stone{8}{6}{black}
		\stone{5}{5}{white}
		\stone{6}{5}{black}
		\stone{7}{5}{black}
        \stone{5}{8}{white}
        
	\end{tikzpicture}
\end{figure}

\noindent\textbf{4-I.} Ход белых.

\begin{figure}[h!]
	\centering
	\begin{tikzpicture}
		\draw[step=1, gray](13.5, 8.5) grid (18, 15.5);
		\draw[black, ultra thick](18, 15.5) -- (18, 8.5);
		%\draw[black, ultra thick](18, 15.5) -- (13.5, 15.5); программа решила, что здесь кончается доска
		\stone{18}{14}{white}
		\stone{16}{13}{black}
		\stone{17}{13}{white}
		\stone{18}{13}{black}
		\stone{16}{12}{black}
		\stone{18}{12}{black}
		\stone{16}{11}{white}
		\stone{17}{11}{white}
		\stone{18}{11}{black}
		\stone{18}{10}{white}
	\end{tikzpicture}
\end{figure}

Можно ли пойти в выделенный пункт?

\noindent\textbf{5-I.} Ход чёрных.

\begin{figure}[h!]
	\centering
	\begin{tikzpicture}
		\draw[step=1, gray](2.5, 0) grid (12.5, 2.75);
		\draw[black, ultra thick](2.5, 0) -- (12.5, 0);
		\stone{5}{2}{black}
		\stone{6}{2}{black}
		\stone{7}{2}{black}
		\stone{8}{2}{black}
		\stone{9}{2}{black}
		\stone{4}{1}{black}
		\stone{6}{1}{white}
		\stone{7}{1}{white}
		\stone{8}{1}{white}
		\stone{9}{1}{white}
		\stone{10}{1}{black}
		\stone{4}{0}{black}
		\stone{5}{0}{white}
		\stone{7}{0}{black}
		\stone{8}{0}{white}
		\stone{9}{0}{white}
		\stone{10}{0}{black}
        \territory{6}{0}{black}
	\end{tikzpicture}
\end{figure}

\noindent\textbf{6-I.} Ход чёрных. В изначальной позиции чёрные сделали ход 1. Могут ли белые ответить в выделенный пункт?


\begin{figure}[h!]
\centering

\begin{subfigure}{0.4\textwidth}
	\centering
	\begin{tikzpicture}
		\draw[step=1, gray](4.5, 6.5) grid (11.5, 13.5);
		\stone{8}{12}{white}
		\stone{9}{12}{white}
		\stone{6}{11}{white}
		\stone{7}{11}{white}
		\stone{9}{11}{white}
		\stone{6}{10}{white}
		\stone{7}{10}{black}
		\stone{8}{10}{white}
		\stone{9}{10}{black}
		\stone{10}{10}{white}
		\stone{6}{9}{white}
		\stone{7}{9}{black}
		\stone{8}{9}{black}
		\stone{9}{9}{black}
		\stone{10}{9}{white}
		\stone{7}{8}{white}
		\stone{8}{8}{white}
		\stone{9}{8}{white}
		\stone{10}{8}{white}
	\end{tikzpicture}
    \caption{Изначальная позиция.}
\end{subfigure}
\hfill
\begin{subfigure}{0.4\textwidth}
	\centering
	\begin{tikzpicture}
		\draw[step=1, gray](4.5, 6.5) grid (11.5, 13.5);
		\stone{8}{12}{white}
		\stone{9}{12}{white}
		\stone{6}{11}{white}
		\stone{7}{11}{white}
		\stone{9}{11}{white}
		\stone{6}{10}{white}
		\stone{7}{10}{black}
		\territory{8}{10}{white}
        \stone{8}{11}{black}
        \stonelabel{8}{11}{white}{$1$}
		\stone{9}{10}{black}
		\stone{10}{10}{white}
		\stone{6}{9}{white}
		\stone{7}{9}{black}
		\stone{8}{9}{black}
		\stone{9}{9}{black}
		\stone{10}{9}{white}
		\stone{7}{8}{white}
		\stone{8}{8}{white}
		\stone{9}{8}{white}
		\stone{10}{8}{white}
	\end{tikzpicture}
    \caption{После хода чёрных.}
\end{subfigure}
\end{figure}


\newpage

\section*{Основы тактики.}\addcontentsline{toc}{section}{Основы тактики}

\subsection*{Тактический элемент №1.}\addcontentsline{toc}{subsection}{Тактический элемент №1}

\subsection*{Тактический элемент №2.}\addcontentsline{toc}{subsection}{Тактический элемент №2}

\subsection*{Тактический элемент №3.}\addcontentsline{toc}{subsection}{Тактический элемент №3}

\newpage

\subsection*{Задачи.}\addcontentsline{toc}{subsection}{Задачи}

\newpage

\section*{Основы стратегии.}\addcontentsline{toc}{section}{Основы стратегии}

\subsection*{Стратегический элемент №1.}\addcontentsline{toc}{subsection}{Стратегический элемент №1}

\subsection*{Стратегический элемент №2.}\addcontentsline{toc}{subsection}{Стратегический элемент №2}

\subsection*{Стратегический элемент №3.}\addcontentsline{toc}{subsection}{Стратегический элемент №3}

\newpage

\subsection*{Задачи.}\addcontentsline{toc}{subsection}{Задачи}

\newpage

\section*{Пояснения и ответы к задачам первой части.}\addcontentsline{toc}{section}{Пояснения и ответы к задачам первой части}
\subsection*{Основы игры.}\addcontentsline{toc}{subsection}{Основы игры}


\noindent\textbf{1-I.}

Группа чёрных камней лишена всех дыханий. Внимание, она не соединена с камнем 1!
\begin{figure}[h!]
\centering
\begin{tikzpicture}
    \draw [step=1,gray] (-0.5,-0.5) grid (7.5,7.5);
    \filldraw[color=black, fill=black!60, very thick](3,3) circle (0.45);
    \filldraw[color=black, fill=black!60, very thick](4,3) circle (0.45);
    \filldraw[color=black, fill=black!60, very thick](2,3) circle (0.45);
    \filldraw[color=black, fill=white, very thick](3,2) circle (0.45);
    \filldraw[color=black, fill=white, very thick](4,2) circle (0.45);
    \filldraw[color=black, fill=white, very thick](2,2) circle (0.45);
    \filldraw[color=black, fill=white, very thick](3,4) circle (0.45);
    \filldraw[color=black, fill=white, very thick](2,4) circle (0.45);
    \filldraw[color=black, fill=white, very thick](4,4) circle (0.45);
    \filldraw[color=black, fill=white, very thick](1,3) circle (0.45);
    \filldraw[color=black, fill=white, very thick](5,3) circle (0.45);
    \filldraw[color=black, fill=black, very thick](3,5) circle (0.45);
    \filldraw[color=black, fill=black, very thick](2,5) circle (0.45);
    \filldraw[color=black, fill=black, very thick](4,5) circle (0.45);
    \filldraw[color=black, fill=black, very thick](5,4) circle (0.45);
    \node[text=white] at (5,4){1};
\end{tikzpicture}
\end{figure}

\noindent\textbf{2-I.}

Ситуация схожа с предыдущей задачей. Только следует обратить внимание на то, что группа окружена не только камнями оппонента, но и краем доски.
\begin{figure}[h!]
	\centering
	\begin{tikzpicture}
		\draw[step=1, gray](0, 0) grid (8.5, 5.5);
		\draw[black, ultra thick](0, 0) -- (0, 5.5);
		\draw[black, ultra thick](0, 0) -- (8.5, 0);
		\stone{1}{3}{white}
		\stone{6}{3}{black}
		\stone{2}{2}{white}
		\stone{3}{2}{white}
		\stone{4}{2}{white}
		\stone{5}{2}{white}
		\stone{6}{2}{black}
		\stone{2}{1}{black}
		\stone{3}{1}{black}
		\stone{4}{1}{black}
		\stone{5}{1}{black}
		\stone{6}{1}{white}
		\stone{7}{1}{white}
		\stone{1}{0}{black}
		\stone{2}{0}{black!15}
		\stone{3}{0}{black!15}
		\stone{4}{0}{black!15}
		\stone{5}{0}{black!15}
		\stone{6}{0}{black}
		\stone{7}{0}{black}
        
	\end{tikzpicture}
\end{figure}


\newpage
\noindent\textbf{3-I.} 

Видно, что у отмеченной группы камней осталось только одно дыхание, поэтому в этот пункт следует сходить чёрным.


\begin{figure}[h!]
	\centering
	\begin{tikzpicture}
		\draw[step=1, gray](2.5, 3.5) grid (9.5, 10.5);
		\stone{7}{9}{white}
		\stone{7}{8}{black}
		\stone{7}{7}{white}
        \stone{5}{8}{white}
		\stone{8}{7}{black}
		\stone{4}{6}{black}
		\stone{5}{6}{black}
		\stone{6}{6}{white}
		\stone{7}{6}{white}
		\stone{8}{6}{black}
		\stone{5}{5}{white}
		\stone{6}{5}{black}
		\stone{7}{5}{black}
        \territory{6}{7}{white}
        \stonelabel{6}{6}{black}{$\times$}
        \stonelabel{7}{6}{black}{$\times$}
        \stonelabel{7}{7}{black}{$\times$}
	\end{tikzpicture}
\end{figure}

\noindent\textbf{4-I.} 

Опять отмеченная группа имеет только одно дыхание.

\begin{figure}[h!]
	\centering
	\begin{tikzpicture}
		\draw[step=1, gray](13.5, 8.5) grid (18, 15.5);
		\draw[black, ultra thick](18, 15.5) -- (18, 8.5);
		\stone{18}{14}{white}
		\stone{16}{13}{black}
		\stone{17}{13}{white}
		\stone{18}{13}{black}
		\stone{16}{12}{black}
		\stone{18}{12}{black}
		\stone{16}{11}{white}
		\stone{17}{11}{white}
		\stone{18}{11}{black}
		\stone{18}{10}{white}
        \stonelabel{18}{11}{white}{$\times$}
        \stonelabel{18}{12}{white}{$\times$}
        \stonelabel{18}{13}{white}{$\times$}
        \territory{17}{12}{black}
	\end{tikzpicture}
\end{figure}

\noindent\textbf{5-I.} 

Такой ход сделать нельзя, т.к. это самоубийственный ход (у этого камня и у камня 1 не останется дыханий). Притом белые камни имеют два дыхания, т.е. они не лишаются этим ходом.

Замечание. В этой позиции можно серией 2-3 ходов снять все белые камни. К данной диаграмме мы ещё вернёмся в следующих главах.

\begin{figure}[h!]
	\centering
	\begin{tikzpicture}
		\draw[step=1, gray](2.5, 0) grid (12.5, 2.75);
		\draw[black, ultra thick](2.5, 0) -- (12.5, 0);
		\stone{5}{2}{black}
		\stone{6}{2}{black}
		\stone{7}{2}{black}
		\stone{8}{2}{black}
		\stone{9}{2}{black}
		\stone{4}{1}{black}
		\stone{6}{1}{white}
		\stone{7}{1}{white}
		\stone{8}{1}{white}
		\stone{9}{1}{white}
		\stone{10}{1}{black}
		\stone{4}{0}{black}
		\stone{5}{0}{white}
		\stone{7}{0}{black}
        \stonelabel{7}{0}{white}{1}
		\stone{8}{0}{white}
		\stone{9}{0}{white}
		\stone{10}{0}{black}
        \territory{6}{0}{black}
        \territory{5}{1}{black}
	\end{tikzpicture}
\end{figure}

\noindent\textbf{6-I.} Догадливый читатель легко догадывается, что в этой задаче спрашивается нание правила ко, коли уточнены несколько предыдущих ходов.




\begin{figure}[h!]
\centering

\begin{subfigure}{0.4\textwidth}
	\centering
	\begin{tikzpicture}
		\draw[step=1, gray](4.5, 6.5) grid (11.5, 13.5);
		\stone{8}{12}{white}
		\stone{9}{12}{white}
		\stone{6}{11}{white}
		\stone{7}{11}{white}
		\stone{9}{11}{white}
		\stone{6}{10}{white}
		\stone{7}{10}{black}
		\stone{8}{10}{white}
		\stone{9}{10}{black}
		\stone{10}{10}{white}
		\stone{6}{9}{white}
		\stone{7}{9}{black}
		\stone{8}{9}{black}
		\stone{9}{9}{black}
		\stone{10}{9}{white}
		\stone{7}{8}{white}
		\stone{8}{8}{white}
		\stone{9}{8}{white}
		\stone{10}{8}{white}
	\end{tikzpicture}
    \caption{Изначальная позиция.}
\end{subfigure}
\hfill
\begin{subfigure}{0.4\textwidth}
	\centering
	\begin{tikzpicture}
		\draw[step=1, gray](4.5, 6.5) grid (11.5, 13.5);
		\stone{8}{12}{white}
		\stone{9}{12}{white}
		\stone{6}{11}{white}
		\stone{7}{11}{white}
		\stone{9}{11}{white}
		\stone{6}{10}{white}
		\stone{7}{10}{black}
		\territory{8}{10}{white}
        \stone{8}{11}{black}
        \stonelabel{8}{11}{white}{$1$}
		\stone{9}{10}{black}
		\stone{10}{10}{white}
		\stone{6}{9}{white}
		\stone{7}{9}{black}
		\stone{8}{9}{black}
		\stone{9}{9}{black}
		\stone{10}{9}{white}
		\stone{7}{8}{white}
		\stone{8}{8}{white}
		\stone{9}{8}{white}
		\stone{10}{8}{white}
	\end{tikzpicture}
    \caption{После хода чёрных.}
\end{subfigure}

\begin{subfigure}{0.4\textwidth}

    \centering
	\begin{tikzpicture}
		\draw[step=1, gray](4.5, 6.5) grid (11.5, 13.5);
		\stone{8}{12}{white}
		\stone{9}{12}{white}
		\stone{6}{11}{white}
		\stone{7}{11}{white}
		\stone{9}{11}{white}
		\stone{6}{10}{white}
		\stone{7}{10}{black!60}
        \stonelabel{7}{10}{white}{$\times$}
		\stone{8}{10}{white}
        \stone{8}{11}{black!60}
		\stone{9}{10}{black!60}
        \stonelabel{9}{10}{white}{$\times$}
		\stone{10}{10}{white}
		\stone{6}{9}{white}
		\stone{7}{9}{black!60}
        \stonelabel{7}{9}{white}{$\times$}
		\stone{8}{9}{black!60}
        \stonelabel{8}{9}{white}{$\times$}
		\stone{9}{9}{black!60}
        \stonelabel{9}{9}{white}{$\times$}
		\stone{10}{9}{white}
		\stone{7}{8}{white}
		\stone{8}{8}{white}
		\stone{9}{8}{white}
		\stone{10}{8}{white}
	\end{tikzpicture}
    \caption{Взятие белых.}
\end{subfigure}
\hfill
\begin{subfigure}{0.4\textwidth}
    \centering
	\begin{tikzpicture}
		\draw[step=1, gray](4.5, 6.5) grid (11.5, 13.5);
		\stone{8}{12}{white}
		\stone{9}{12}{white}
		\stone{6}{11}{white}
		\stone{7}{11}{white}
		\stone{9}{11}{white}
		\stone{6}{10}{white}
		\stone{8}{10}{white}
		\stone{10}{10}{white}
		\stone{6}{9}{white}
		\stone{10}{9}{white}
		\stone{7}{8}{white}
		\stone{8}{8}{white}
		\stone{9}{8}{white}
		\stone{10}{8}{white}
	\end{tikzpicture}
    \caption{После взятия белых.}
\end{subfigure}
\end{figure}

Поначалу некоторым может показаться, что, совершив ответное взятие, белые нарушат правило ко, однако заметим, что выделенные крестиком камни тоже должны уйти с доски после хода белых, поэтому позиция не повторится. (Конечно, подразумевается в условии задачи, что до рассматриваемой ситуации на доске токой конфигурации камней ещё не было.)

\newpage

\subsection*{Основы тактики.}\addcontentsline{toc}{subsection}{Основы тактики}

\newpage

\subsection*{Основы стратегии.}\addcontentsline{toc}{subsection}{Основы стратегии}

\newpage

\section*{Ответы к задачам второй части.}\addcontentsline{toc}{section}{Ответы к задачам второй части}
\subsection*{Основы игры.}\addcontentsline{toc}{subsection}{Основы игры}

\newpage

\subsection*{Основы тактики.}\addcontentsline{toc}{subsection}{Основы тактики}

\newpage

\subsection*{Основы стратегии.}\addcontentsline{toc}{subsection}{Основы стратегии}
\end{document}
