\documentclass[14pt,a4paper]{extarticle}
\usepackage[utf8]{inputenc}
\usepackage[T2A]{fontenc}
\usepackage[english,russian]{babel}
\usepackage{amsmath,amsfonts,amssymb,amsthm,mathtools,tikz,subcaption,float}
\setlength{\topmargin}{-1.5cm}
\setlength{\textheight}{23.3cm}
\setlength{\oddsidemargin}{-0.5cm}
\setlength{\textwidth}{17.0cm}
\newcommand{\V}[1]{\mbox{\boldmath$#1$}}
\newcommand{\T}[1]{\mbox{\bf#1}}
\DeclareUnicodeCharacter{2061}{}
\usepackage{graphicx}
\graphicspath{ {./images} }
\usepackage{wrapfig}
\textheight=240mm
\usepackage{mathtools}
\pagestyle{empty}
\setlength\parindent{12.5mm}
\usepackage{tocloft}
\renewcommand{\cfttoctitlefont}{\normalsize\bfseries}
\renewcommand{\cftsecleader}{\cftdotfill{\cftdotsep}}%простые разреженные отточия
\renewcommand{\cftsecfont}{\normalsize}
\renewcommand\cftsecpagefont{\normalfont}
\renewcommand{\baselinestretch}{1.25}
\usepackage{titlesec}
\titleformat{\section}
  {\normalfont\normalsize\bfseries}{\thesection}{1em}{}
\titleformat{\subsection}
  {\normalfont\normalsize\bfseries}{\thesection}{1em}{}
\usepackage{tocloft}
\renewcommand{\cfttoctitlefont}{\normalsize\bfseries}
\renewcommand{\cftsecleader}{\cftdotfill{\cftdotsep}}%простые разреженные отточия
\renewcommand{\cftsecfont}{\normalsize}
\renewcommand\cftsecpagefont{\normalfont}


%%%%%%%%%%%%%%%%% Го библиотека
\newcommand{\stone}[3]{\filldraw[color=black, fill=#3, very thick](#1,#2) circle(0.45);}
\newcommand{\territory}[3]{\filldraw[color=black, fill=#3, very thick](#1, #2) circle(0.125);}
\newcommand{\stonelabel}[4]{\node[text=#3] at (#1,#2) {#4};}

\newenvironment{goboard}[2][7.5] {
    \begin{figure}
    \centering
    \begin{tikzpicture}
    \draw [step=1, gray] (#2, #2) grid(#1, #1);
} {
    \end{tikzpicture}
    \end{figure}
}

\newenvironment{gocorner}[1][4.5]{
    \begin{figure}
    \centering
    \begin{tikzpicture}
    \draw[step=1, gray] (0, 0) grid(#1, #1);
    \draw[black,ultra thick](0,0) -- (#1, 0);
    \draw[black,ultra thick](0,0) -- (0, #1);
}{
    \end{tikzpicture}
    \end{figure}
}
%%%%%%%%%%%%%%%%% Го библиотека конец

\begin{document}

\thispagestyle{empty}
\begin{center}
\small{МИНИСТЕРСТВО НАУКИ И ВЫСШЕГО ОБРАЗОВАНИЯ}\par
\small{РОССИЙСКОЙ ФЕДЕРАЦИИ}\par
\small{ФЕДЕРАЛЬНОЕ ГОСУДАРСТВЕННОЕ АВТОНОМНОЕ ОБРАЗОВАТЕЛЬНОЕ УЧРЕЖДЕНИЕ ВЫСШЕГО ОБРАЗОВАНИЯ}\par
\small{<<ЮЖНЫЙ ФЕДЕРАЛЬНЫЙ УНИВЕРСИТЕТ>>}\par
\normalsize{Институт математики механики и компьютерных наук им. И.\,И. Воровича}\par
\end{center}


\vspace*{5.0em plus 0.2em minus 0.2em}

\begin{center}
Методичка по игре в го\\
для начинающих.
\end{center}

\begin{center}
Часть 1?
\end{center}

\vspace*{5.0em plus 0.2em minus 0.2em}
\begin{flushright}
 Выполнил(и):\\
	...
\end{flushright}


\vspace*{\fill}

\begin{center}
г. Ростов-на-Дону, 2023
\end{center}


%%%%%%%%%%%%%%%%%
\newpage
%%%%%%%%%%%%%%%%%

\thispagestyle{empty}
\tableofcontents


%%%%%%%%%%%%%%%%%
\newpage
%%%%%%%%%%%%%%%%%


\section*{Предисловие.}\addcontentsline{toc}{section}{Предисловие}


%%%%%%%%%%%%%%%%%
\newpage
%%%%%%%%%%%%%%%%%


\section*{Основы игры.}\addcontentsline{toc}{section}{Основы игры}

\subsection*{Правила игры.}\addcontentsline{toc}{subsection}{Правила игры}

Для игры понадобится доска $9\times9$, $13\times13$, $15\times15$, или $19\times19$ (доски поменьше подходят для начинающих игроков) и два набора камней разных цветов (обычно чёрного и белого).

Рассмотрим 5 базовых правил го:\\

\textit{Правило 1.} О постановке камней на доску.

Чёрные и белые ходят по очереди. Начинают партию чёрные с пустой доски. Каждым своим ходом игроки могут поставить камень своего цвета на доску или сказать "пас". Камни ставятся на пересечение линий, называемое пунктом. Уже поставленные камни нельзя передвигать (однако, их может снять противник, о чём будет рассказано немного позднее).

\begin{figure}[h]
    \centering
    \begin{tikzpicture}
        \draw[step=1,gray] (0,0) grid (5.5,5.5);
        \draw[black,ultra thick](0,0) -- (5.5,0);
        \draw[black,ultra thick](0,0) -- (0,5.5);
        \stone{3}{4}{black}
        \stone{1}{2}{white}
        \stone{1}{1}{black}
        \stone{4}{2}{white}
        \stone{4}{0}{black}
    \end{tikzpicture}
    \caption{Просто какая-то расстановка камней на фрагменте доски.} \label{1}
\end{figure}

\textit{Правило 2.} О дыханиях и снятии камней с доски.

Под дыханиями (или степенями свободы, или дамэ) камня будем понимать все соседние с ним пункты, не занятые другими камнями. Внимание: пункты, расположенные от камня по диагонали, не считаются соседними с ним!


%%%%%%%%%%%%%%%%%
\newpage
%%%%%%%%%%%%%%%%%


\begin{figure}[h!]
    \centering
    \begin{tikzpicture}
        \draw [step=1,gray] (0,0) grid (5.5,5.45);
        \draw[black,ultra thick](0,0) -- (5.5,0);
        \draw[black,ultra thick](0,0) -- (0,5.5);
        \stone{0}{0}{black}
        \territory{1}{0}{black}
        \territory{0}{1}{black}
        \stone{1}{4}{white}
        \stone{0}{4}{black}
        \territory{0}{5}{black}
        \territory{0}{3}{black}
        \stone{3}{3}{black}
        \territory{2}{3}{black}
        \territory{4}{3}{black}
        \territory{3}{2}{black}
    \territory{3}{4}{black}
    \end{tikzpicture}
    \caption{Дыхания камней.} \label{2}
\end{figure}

Говорят, что камни одного цвета образуют группу камней, если они соединены по стороне. У группы камней дыхания общие.

\begin{figure}[h!]
    \centering
    
    \begin{subfigure}[t]{0.35\textwidth}
    \begin{tikzpicture}
        \draw [step=1,gray] (0,0) grid (5.5,5.45);
        \draw[black,ultra thick](0,0) -- (5.5,0);
        \draw[black,ultra thick](0,0) -- (0,5.5);
        \stone{3}{3}{black}
        \stone{4}{3}{black}
        \stone{2}{3}{black}
        \territory{3}{2}{black}
        \territory{4}{2}{black}
        \territory{2}{2}{black}
        \territory{3}{4}{black}
        \territory{2}{4}{black}
        \territory{4}{4}{black}
        \territory{1}{3}{black}
        \territory{5}{3}{black}
    \end{tikzpicture}
    \caption{Дыхания группы камней.} 
    \label{3a}
    \end{subfigure}
    \hfill
    \begin{subfigure}[t]{0.32\textwidth}
    \begin{tikzpicture}
        \draw [step=1,gray] (0,0) grid (4.5,5.5);
        \draw[black,ultra thick](0,0) -- (4.5,0);
        \draw[black,ultra thick](0,0) -- (0,5.5);
        \stone{3}{0}{black}
        \stone{4}{0}{black}
        \stone{3}{1}{black}
        \territory{2}{0}{black}
        \territory{2}{1}{black}
        \territory{3}{2}{black}
        \territory{4}{1}{black}
        \territory{5}{0}{black}
    \end{tikzpicture}
    \caption{Дыхания группы камней.} 
    \label{3b}
    \end{subfigure}
    \hfill
    \begin{subfigure}[t]{0.27\textwidth}
    \begin{tikzpicture}
        \draw [step=1,gray] (0,0) grid (3.5,5.5);
        \draw[black,ultra thick](0,0) -- (3.5,0);
        \draw[black,ultra thick](0,0) -- (0,5.5);
        \stone{3}{1}{black}
        \stone{4}{2}{black}
        \stone{2}{2}{black}
    \end{tikzpicture}
    \caption{Не группа камней! Это просто 3 камня.} 
    \label{3c}
    \end{subfigure}
    \caption{}
\end{figure}

Если камень или группа камней лишаетяся всех дыханий, то эти камни снимаются с доски и попадают во вражеский плен.


%%%%%%%%%%%%%%%%%
\newpage
%%%%%%%%%%%%%%%%%


\begin{figure}[h!]
    \centering
    \begin{subfigure}[t]{0.4\textwidth}
    \begin{tikzpicture}
        \draw [step=1,gray] (0,0) grid (5.5,5.5);
        \draw[black,ultra thick](0,0) -- (5.5,0);
        \draw[black,ultra thick](0,0) -- (0,5.5);
        \stone{3}{3}{black!60}
        \stone{2}{3}{white}
        \stone{4}{3}{white}
        \stone{3}{2}{white}
        \stone{3}{4}{white}
    \end{tikzpicture}
    \caption{}
    \label{4a}
    \end{subfigure}
    \hfill
    \begin{subfigure}[t]{0.4\textwidth}
        \begin{tikzpicture}
        \draw[step=1,gray] (0,0) grid (5.5,5.5);
        \draw[black,ultra thick](0,0) -- (5.5,0);
        \draw[black,ultra thick](0,0) -- (0,5.5);
        \stone{3}{2}{black!60}
        \stone{3}{3}{black!60}
        \stone{3}{4}{black!60}
        \stone{3}{2}{white}
        \stone{4}{2}{white}
        \stone{2}{2}{white}
        \stone{3}{4}{white}
        \stone{2}{4}{white}
        \stone{4}{4}{white}
        \stone{1}{3}{white}
        \stone{5}{3}{white}
        \end{tikzpicture}
        \caption{}
        \label{4b}
    \end{subfigure}
    \caption{Убитые камни или группы камней.}
    \label{4}
\end{figure}

\textit{Правило 3.} О запрещённом ходе и исключении и этого правила.

Представим, какая позиция осталась на рис \ref{4a} после снятия чёрного камня. Поймём, что если вдруг чёрные захотят поставить камень на то же место, то он уже будет лишён всех дыханий.
Согласно правилам го, запрещено делать самоубийсвенный ход.

\begin{figure}[h!]
    \centering
    \begin{tikzpicture}
        \draw [step=1,gray] (0,0) grid (5.5,5.5);
        \draw[black,ultra thick](0,0) -- (5.5,0);
        \draw[black,ultra thick](0,0) -- (0,5.5);
        \territory{3}{3}{black}
        \stone{2}{3}{white}
        \stone{4}{3}{white}
        \stone{3}{2}{white}
        \stone{3}{4}{white}
    \end{tikzpicture}
    \caption{Пойдя в выделенный пункт, чёрные сразу лишат себя дыханий.}
    \label{5}
\end{figure}

К тому же запрещено делать ход, отбирающий у своей группы посленднюю степень свободы. На рис. \ref{6} при попытке чёрных поёти в выделенный пункт, образуется позиция, как на рис. \ref{4b}, что, как мы видим, лишает группу всех оставшихся чтепеней свободы.


%%%%%%%%%%%%%%%%%
\newpage
%%%%%%%%%%%%%%%%%


\begin{figure}[h!]
\centering
\begin{tikzpicture}
\draw [step=1,gray] (0,0) grid (5.5,5.5);
    \draw[black,ultra thick](0,0) -- (5.5,0);
    \draw[black,ultra thick](0,0) -- (0,5.5);
    \stone{3}{3}{black}
    \stone{4}{3}{black}
    \territory{2}{3}{black}
    \stone{3}{2}{white}
    \stone{4}{2}{white}
    \stone{2}{2}{white}
    \stone{3}{4}{white}
    \stone{2}{4}{white}
    \stone{4}{4}{white}
    \stone{1}{3}{white}
    \stone{5}{3}{white}
\end{tikzpicture}

\caption{Пойдя в выделенный пункт, чёрные отнимут последнее дыхание у группы.}
\label{6}
\end{figure}
Теперь предположим, что в процессе игры, изображённой на рис. \ref{5} позиция преобразовалась к диаграмме \ref{7}. В этой, по правилам, уже можно сделать ход чёрным в выделенный пункт, т.к. белый камень 1 остаются без единой степени свободы. Иначе говоря, самоубийсвенный ход делать можно, если притом вражеский(е) камень(камни) лишаются последней степени свободы. И т.к. у поставленного чёрного камня уже будет дыхание, то он (пока что) будет поствлен.

\begin{figure}[h!]
    \centering
    \begin{tikzpicture}
        \draw [step=1,gray] (0,0) grid (5.5,5.5);
        \draw[black,ultra thick](0,0) -- (5.5,0);
        \draw[black,ultra thick](0,0) -- (0,5.5);
        \territory{3}{3}{black}
        \stone{2}{3}{white}
        \stonelabel{3}{2}{black}{1}
        \stone{4}{3}{white}
        \stone{3}{2}{white}
        \stone{3}{4}{white}
        \stone{2}{2}{black}
        \stone{1}{3}{black}
        \stone{2}{4}{black}
    \end{tikzpicture}
    \caption{Пойдя в выделенный пункт, чёрные лишат сначала белый камень 1 дыхания.}
    \label{7}
\end{figure}

\textit{Правило 4.} Правило ко.

Пусть в позиции, изображённой на рис. \ref{7}, чёрные взяли белый камень, тогда заметим, что у этого чёрного камня осталось одно дыхание, и белые могут вернуть сразу же камень назад. И мы вновь вернулись к начальной позиции.

\begin{figure}[h!]
    \centering
    \begin{subfigure}{0.4\textwidth}
    \begin{tikzpicture}
        \draw[step=1,gray] (0,0) grid (7.5,7.5);
        \draw[black,ultra thick](0,0) -- (7.5,0);
        \draw[black,ultra thick](0,0) -- (0,7.5);
        \stone{3}{3}{black}
        \stonelabel{3}{3}{white}{1}
        \stone{4}{3}{white}
        \stone{3}{2}{white}
        \stone{3}{4}{white}
        \stone{2}{2}{black}
        \stone{1}{3}{black}
        \stone{2}{4}{black}
    \end{tikzpicture}
    \end{subfigure}
    \hfill
    \begin{subfigure}{0.4\textwidth}
    \begin{tikzpicture}
        \draw [step=1,gray] (0,0) grid (7.5,7.5);
        \draw[black,ultra thick](0,0) -- (7.5,0);
        \draw[black,ultra thick](0,0) -- (0,7.5);
        \stone{2}{3}{white}
        \stonelabel{2}{3}{black}{2}
        \stone{4}{3}{white}
        \stone{3}{2}{white}
        \stone{3}{4}{white}
        \stone{2}{2}{black}
        \stone{1}{3}{black}
        \stone{2}{4}{black}
    \end{tikzpicture}
    \end{subfigure}
    \caption{Повторение позиции.}
    \label{8}
\end{figure}

Для исключения бесконечного повторения позиции существует правило ко.
Согласно ему, никакая позиция на доске в партии после непасового хода не может повториться.

Например, в приведённой позиции партия модет пойти следующим образом, как на рис. \ref{9}

\begin{figure}[H]
    \centering
    
    \begin{subfigure}[t]{0.4\textwidth}
    \begin{tikzpicture}[xscale=0.8,yscale=0.8]
        \draw [step=1,gray] (0,0) grid (7.5,7.5);
        \draw[black,ultra thick](0,0) -- (7.5,0);
        \draw[black,ultra thick](0,0) -- (0,7.5);
        \stone{3}{3}{black}
        \stonelabel{3}{3}{white}{1}
        \stone{4}{3}{white}
        \stone{3}{2}{white}
        \stone{3}{4}{white}
        \stone{2}{2}{black}
        \stone{1}{3}{black}
        \stone{2}{4}{black}
    \end{tikzpicture}
    \caption{Чёрные сняли камень.}
    \label{9a}
    \end{subfigure}
    \hfill
    \begin{subfigure}[t]{0.4\textwidth}
    \begin{tikzpicture}[xscale=0.8,yscale=0.8]
        \draw [step=1,gray] (0,0) grid (7.5,7.5);
        \draw[black,ultra thick](0,0) -- (7.5,0);
        \draw[black,ultra thick](0,0) -- (0,7.5);
        \stone{3}{3}{black}
        \stone{6}{6}{white}
        \stonelabel{6}{6}{black}{2}
        \stone{4}{3}{white}
        \stone{3}{2}{white}
        \stone{3}{4}{white}
        \stone{2}{2}{black}
        \stone{1}{3}{black}
        \stone{2}{4}{black}
    \end{tikzpicture}
    \caption{Белые не могут снять в ответ, поэтому пошли в другое место.}
    \label{9b}
    \end{subfigure}
    
    \begin{subfigure}[t]{0.4\textwidth}
    \begin{tikzpicture}[xscale=0.8,yscale=0.8]
        \draw [step=1,gray] (0,0) grid (7.5,7.5);
        \draw[black,ultra thick](0,0) -- (7.5,0);
        \draw[black,ultra thick](0,0) -- (0,7.5);
        \stone{3}{3}{black}
        \stone{6}{6}{white}
        \stone{5}{7}{black}
        \stonelabel{5}{7}{white}{3}
        \stone{4}{3}{white}
        \stone{3}{2}{white}
        \stone{3}{4}{white}
        \stone{2}{2}{black}
        \stone{1}{3}{black}
        \stone{2}{4}{black}
    \end{tikzpicture}
    \caption{Пусть чёрным необходимо ответить на ход 2.}
    \label{9c}
    \end{subfigure}
    \hfill
    \begin{subfigure}[t]{0.4\textwidth}
    \begin{tikzpicture}[xscale=0.8,yscale=0.8]
        \draw [step=1,gray] (0,0) grid (7.5,7.5);
        \draw[black,ultra thick](0,0) -- (7.5,0);
        \draw[black,ultra thick](0,0) -- (0,7.5);
        \stone{2}{3}{white}
        \stone{6}{6}{white}
        \stone{5}{7}{black}
        \stonelabel{3}{2}{black}{4}
        \stone{4}{3}{white}
        \stone{3}{2}{white}
        \stone{3}{4}{white}
        \stone{2}{2}{black}
        \stone{1}{3}{black}
        \stone{2}{4}{black}
    \end{tikzpicture}
    \caption{Теперь белые могут снять камень, т.к. позиция изменилась.}
    \label{9d}
    \end{subfigure}
    
    \begin{subfigure}[t]{0.4\textwidth}
    \begin{tikzpicture}[xscale=0.8,yscale=0.8]
        \draw [step=1,gray] (0,0) grid (7.5,7.5);
        \draw[black,ultra thick](0,0) -- (7.5,0);
        \draw[black,ultra thick](0,0) -- (0,7.5);
        \stone{2}{3}{white}
        \stone{6}{6}{white}
        \stone{5}{7}{black}
        \stone{5}{6}{black}
        \stonelabel{5}{6}{white}{5}
        \stone{4}{3}{white}
        \stone{3}{2}{white}
        \stone{3}{4}{white}
        \stone{2}{2}{black}
        \stone{1}{3}{black}
        \stone{2}{4}{black}
    \end{tikzpicture}
    \caption{Чёрные не могут снять в ответ, поэтому пошли в другое место.}
    \label{9e}
    \end{subfigure}
    \hfill
    \begin{subfigure}[t]{0.4\textwidth}
    \begin{tikzpicture}[xscale=0.8,yscale=0.8]
        \draw [step=1,gray] (0,0) grid (7.5,7.5);
        \draw[black,ultra thick](0,0) -- (7.5,0);
        \draw[black,ultra thick](0,0) -- (0,7.5);
        \stone{2}{3}{white}
        \stone{6}{6}{white}
        \stone{5}{7}{black}
        \stone{5}{6}{black}
        \stone{3}{3}{white}
        \stonelabel{3}{3}{black}{6}
        \stone{4}{3}{white}
        \stone{3}{2}{white}
        \stone{3}{4}{white}
        \stone{2}{2}{black}
        \stone{1}{3}{black}
        \stone{2}{4}{black}
    \end{tikzpicture}
    \caption{Белые решили соединить камни, не среагировав на ход 5.}
    \label{9f}
    \end{subfigure}
    \caption{Ко-борьба.}
    \label{9}
\end{figure}

Здесь правило ко опрелелено неклассическим образом, в виду некоторых сложных случаев, которые пока что рассматриваться не будут.

\textit{Правило 5.} Территория, конец партии и подсчёт очков.

Партия заканчивается, когда оба игрока по очереди спасуют. Это может произойти, когда у обоих игроков не осталось ходов, приносящих очков.

Под территорией будем понимать все не занятые камнями пункты, полностью окружённые камнями одного цвета. Это определение дано не полгостью и включает в себя ещё ряд важных аспектов, о которых будет сказано при решении задач. По большей части понятие территории являтся договорным, т.е. если оба игрока согласны, что определённая часть доски является территорией одного игрока, то так и будет считаться в итоговом результате. Пока что достаточно интуитивного понимания территории.

\begin{figure}[h]
\centering
\begin{tikzpicture}
    \draw[step=1,gray] (1,1) grid (11.5,11.5);
    \draw[black,ultra thick](1,1) -- (11.5,1);
    \draw[black,ultra thick](1,1) -- (1,11.5);
    \stone{1}{1}{black}
    \stone{2}{2}{black}
    \stone{3}{3}{black}
    \stone{2}{3}{black}
    \stone{3}{1}{black}
    \stone{3}{2}{black}
    \stone{1}{3}{black}
    \territory{2}{1}{black}
    \territory{1}{2}{black}

    \stone{6}{3}{white}
    \stone{8}{3}{white}
    \stone{7}{2}{white}
    \stone{7}{4}{white}
    \territory{7}{3}{white}

    \territory{4}{6}{white}
    \territory{5}{6}{white}
    \territory{3}{6}{white}
    \stone{4}{5}{white}
    \stone{5}{5}{white}
    \stone{3}{5}{white}
    \stone{4}{7}{white}
    \stone{3}{7}{white}
    \stone{5}{7}{white}
    \stone{2}{6}{white}
    \stone{6}{6}{white}

    \stone{9}{7}{black}
    \stone{10}{7}{black}
    \stone{8}{7}{black}
    \stone{10}{8}{black}
    \stone{10}{9}{black}
    \stone{8}{8}{black}
    \stone{7}{8}{black}
    \stone{6}{10}{black}
    \stone{6}{8}{black}
    \stone{6}{9}{black}
    \stone{7}{10}{black}
    \stone{8}{10}{black}
    \stone{9}{10}{black}
    \stone{10}{10}{black}
    \territory{9}{8}{black}
    \territory{9}{9}{black}
    \territory{8}{9}{black}
    \territory{7}{9}{black}
\end{tikzpicture}
\caption{Территории.}
\label{10}
\end{figure}


%%%%%%%%%%%%%%%%%
\newpage
%%%%%%%%%%%%%%%%%


\subsection*{Задачи.}\addcontentsline{toc}{subsection}{Задачи}

\


Определить какие группы мертвы, т.е. у них не осталось степеней свободы.


\noindent\textbf{1-I.}
\begin{goboard}{-0.5}
    \stone{3}{3}{black}
    \stone{4}{3}{black}
    \stone{2}{3}{black}
    \stone{3}{2}{white}
    \stone{4}{2}{white}
    \stone{2}{2}{white}
    \stone{3}{4}{white}
    \stone{2}{4}{white}
    \stone{4}{4}{white}
    \stone{1}{3}{white}
    \stone{5}{3}{white}
    \stone{3}{5}{black}
    \stone{2}{5}{black}
    \stone{4}{5}{black}
    \stone{5}{4}{black}
\end{goboard}


%%%%%%%%%%%%%%%%%
\newpage
%%%%%%%%%%%%%%%%%


\subsection*{Пояснения и ответы к задачам первой части.}\addcontentsline{toc}{subsection}{Пояснения и ответы к задачам первой части}

\noindent\textbf{1-I.}
Группа чёрных камней лишена всех дыханий. Внимание, она не соединена с камнем 1!
\noindent\textbf{1-I.}
\begin{goboard}{-0.5}
    \stone{3}{3}{black!60}
    \stone{4}{3}{black!60}
    \stone{2}{3}{black!60}
    \stone{3}{2}{white}
    \stone{4}{2}{white}
    \stone{2}{2}{white}
    \stone{3}{4}{white}
    \stone{2}{4}{white}
    \stone{4}{4}{white}
    \stone{1}{3}{white}
    \stone{5}{3}{white}
    \stone{3}{5}{black}
    \territory{3}{6}{black}
    \territory{5}{6}{white}
    \stone{4}{5}{black}
    \stone{5}{4}{black}
    \stone{2}{5}{black}
    \stonelabel{2}{5}{white}{1}
    \stonelabel{3}{5}{white}{2}
    \stonelabel{4}{5}{white}{3}
\end{goboard}


%%%%%%%%%%%%%%%%%
\newpage
%%%%%%%%%%%%%%%%%


\section*{Базовая тактика и стратегия}\addcontentsline{toc}{section}{Базовая тактика и стратегия}
\section*{Ответы к задачам второй части.}\addcontentsline{toc}{section}{Ответы к задачам второй части}
\end{document}
